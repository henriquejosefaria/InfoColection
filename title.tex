

\title{Threat Model for a Precision Agriculture System}

\author{Henrique Faria A82200 \and Paulo Rosa A81139}


\institute{Departamento de Informática, Universidade do Minho}


\maketitle

% Abstract
\begin{abstract}
Este trabalho tem como objetivo reportar as informações recolhidas no âmbito do estudo de duas empresas, uma de grandes dimensões e uma de pequenas dimensões.\newline
Para a realização da recolha de informações foram feitas pesquisas no google sobre a empresas visada, consultaram-se on respetivos portais na internet e descarregu-se com recurso ao comando \textit{wget} as respetivas páginas sendo estas analisadas visando revelar informações guardadas como anotações, e fizeram-se querys com os comandos \textit{dig} e \textit{whois} acerca dos servidores responsáveis pela footprint da empresa na internet. Para alé disso pode também ser consultado o site \href{https://web.archive.org} para obter informações que digam respeito ás tecnologias usadas em páginas antigas ou ao próprio código modificado.\newline  
Também se recorreu ao uso de ferramentas como o \textit{open port finder} online, o comando host e o zenMap para auxiliar o footprinting das empresas.
Neste são especificados os tipos de dados que se podem pretender obter com um ataque á empresa e como estes são tratados. Também são referidos contactos de telemóvel, telefone, emails e localizações consideradas pertinentes. No fim exibem-se querys feitas aos servidores e as respostas obtidas sendo prossecutidas por uma breve explicação. 

\keywords{Wget \and Whois \and Dig \and Open Port Finder \and Host \and ZenMap \and Pingo Doce \and Gamingreplay}
\end{abstract}
\newpage
